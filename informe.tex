\documentclass[]{article}
\usepackage{lmodern}
\usepackage{amssymb,amsmath}
\usepackage{ifxetex,ifluatex}
\usepackage{fixltx2e} % provides \textsubscript
\ifnum 0\ifxetex 1\fi\ifluatex 1\fi=0 % if pdftex
  \usepackage[T1]{fontenc}
  \usepackage[utf8]{inputenc}
\else % if luatex or xelatex
  \ifxetex
    \usepackage{mathspec}
  \else
    \usepackage{fontspec}
  \fi
  \defaultfontfeatures{Ligatures=TeX,Scale=MatchLowercase}
\fi
% use upquote if available, for straight quotes in verbatim environments
\IfFileExists{upquote.sty}{\usepackage{upquote}}{}
% use microtype if available
\IfFileExists{microtype.sty}{%
\usepackage{microtype}
\UseMicrotypeSet[protrusion]{basicmath} % disable protrusion for tt fonts
}{}
\usepackage[margin=1in]{geometry}
\usepackage{hyperref}
\hypersetup{unicode=true,
            pdftitle={TP4 Análisis Exploratorio de Datos},
            pdfauthor={Juan Manuel Berros, Octavio M. Duarte y Gonzalo Barrera Borla},
            pdfborder={0 0 0},
            breaklinks=true}
\urlstyle{same}  % don't use monospace font for urls
\usepackage{graphicx,grffile}
\makeatletter
\def\maxwidth{\ifdim\Gin@nat@width>\linewidth\linewidth\else\Gin@nat@width\fi}
\def\maxheight{\ifdim\Gin@nat@height>\textheight\textheight\else\Gin@nat@height\fi}
\makeatother
% Scale images if necessary, so that they will not overflow the page
% margins by default, and it is still possible to overwrite the defaults
% using explicit options in \includegraphics[width, height, ...]{}
\setkeys{Gin}{width=\maxwidth,height=\maxheight,keepaspectratio}
\IfFileExists{parskip.sty}{%
\usepackage{parskip}
}{% else
\setlength{\parindent}{0pt}
\setlength{\parskip}{6pt plus 2pt minus 1pt}
}
\setlength{\emergencystretch}{3em}  % prevent overfull lines
\providecommand{\tightlist}{%
  \setlength{\itemsep}{0pt}\setlength{\parskip}{0pt}}
\setcounter{secnumdepth}{0}
% Redefines (sub)paragraphs to behave more like sections
\ifx\paragraph\undefined\else
\let\oldparagraph\paragraph
\renewcommand{\paragraph}[1]{\oldparagraph{#1}\mbox{}}
\fi
\ifx\subparagraph\undefined\else
\let\oldsubparagraph\subparagraph
\renewcommand{\subparagraph}[1]{\oldsubparagraph{#1}\mbox{}}
\fi

%%% Use protect on footnotes to avoid problems with footnotes in titles
\let\rmarkdownfootnote\footnote%
\def\footnote{\protect\rmarkdownfootnote}

%%% Change title format to be more compact
\usepackage{titling}

% Create subtitle command for use in maketitle
\newcommand{\subtitle}[1]{
  \posttitle{
    \begin{center}\large#1\end{center}
    }
}

\setlength{\droptitle}{-2em}
  \title{TP4 Análisis Exploratorio de Datos}
  \pretitle{\vspace{\droptitle}\centering\huge}
  \posttitle{\par}
  \author{Juan Manuel Berros, Octavio M. Duarte y Gonzalo Barrera Borla}
  \preauthor{\centering\large\emph}
  \postauthor{\par}
  \predate{\centering\large\emph}
  \postdate{\par}
  \date{16 de Mayo de 2018}


\begin{document}
\maketitle

\begin{verbatim}
## -- Attaching packages ---------------------------------- tidyverse 1.2.1 --
\end{verbatim}

\begin{verbatim}
## √ ggplot2 2.2.1     √ purrr   0.2.4
## √ tibble  1.4.2     √ dplyr   0.7.4
## √ tidyr   0.8.0     √ stringr 1.3.0
## √ readr   1.1.1     √ forcats 0.3.0
\end{verbatim}

\begin{verbatim}
## -- Conflicts ------------------------------------- tidyverse_conflicts() --
## x dplyr::filter() masks stats::filter()
## x dplyr::lag()    masks stats::lag()
\end{verbatim}

\hypertarget{motivo-y-estructura-del-informe}{%
\section{Motivo y Estructura del
Informe}\label{motivo-y-estructura-del-informe}}

En esta ocasión, nuestro objetivo es fundamentalmente lograr
familiaridad con una serie de algoritmos de clasificación. Estos son de
naturaleza y metodología bastante dispar y por lo tanto su comparación
es difícil y quizás requeriría para llevarse a cabo correctamente la
aplicación sobre varios conjuntos de datos. La entrega que estamos
llevando a cabo no sólo incluye programaas que implementan cada uno de
los algoritmos en cuestión si no que procuramos implementar un pequeño
entorno de clasificación. Las funciones comparten una interface común
que permite llamarlas en forma automatizada, cosa que aprovechamos para
hacer una prueba extensiva e intensiva. Pudimnos probar el rendimiento
sobre todas las formulas concebibles sobre los datos disponibles con
objeto de realizar comparaciones entre algoritmos similares y de
distintas familias.

\hypertarget{algoritmos-implementados-detalles-de-la-implementacion-y-algunos-comentarios-teoricos}{%
\section{Algoritmos Implementados: Detalles de la Implementación y
Algunos Comentarios
Teóricos}\label{algoritmos-implementados-detalles-de-la-implementacion-y-algunos-comentarios-teoricos}}

\hypertarget{metodologia-general}{%
\subsection{Metodología General}\label{metodologia-general}}

Para poder aplicar masivamente los algoritmos, pretendimos lograr un
esquema similar al que propone \emph{R} con la función \texttt{lm}.
Todos los modelos disponen de una función que, más allá de los
vericuetos específicos a cada método, es capaz de tomar una tabla de
datos adecuada y una formula (similar a la llamada
\texttt{lm(df,y\textasciitilde{}x)} ), y entrenar una función de
clasificación capaz de tomar como argumento una nueva tabla con más
datos. Por ser significativa la cantidad de código requerida por cada
función, esta vez organizamos todo el TP como un proyecto de mayor
envergadura y separamos el código puro en una carpeta \texttt{R}. En la
carpeta \texttt{vignettes} hay tres archivos importantes: un guión de
lectura y organización de la tabla de datos, otro que realiza y tabula
las clasificaciones según todos los modelos y este archivo de informe.\\
Todas las funciones están diseñadas para tomar formulas multivariadas,
con las predicciones sobre una sola variable como caso particular. El
indicador de eficiencia que elegimos en esta ocasión es la tasa de
acierto promedio durante la evaluación en k pliegues (trabajamos con
\(k=10\)), que en la tabla se llama \textbf{tasa pesada}.\\
Presentamos la tabla con la salida de la evaluación general, ordenada
según la tasa de acierto promedio por pliegue antes de reflexionar sobre
cada algoritmo por separado.

\texttt{\{r\}\ kable(\ head\ (\ tabla\_resultadosc\ )\ \ )}

\hypertarget{regresion-logistica}{%
\subsection{Regresión Logística:}\label{regresion-logistica}}

Para implementar este algoritmo, fue necesario resolver varios
problemas:

\begin{itemize}
\item
  Para parametrizar el modelo se requiere una función de descenso por el
  gradiente (\texttt{dg}, en el archivo \texttt{rlog.R}).
\item
  La función de penalización (maximización) asociada es la
  logverosimilitud, que también fue implementada junto a su gradiente
  para poder aplicar el proceso de maximización.
\item
  Dado que esta función nos devuelve en realidad las probabilidades de
  pertenecer a la clase asociada con el valor 1, es posible implementar
  también la curva ROC.

  Si bien el manual no hace énfasis en la posibilidad de emplear la
  regresión logística sobre modelos con más de un predictor, en clase
  vimos el caso general para n predictores y por esta razón decidimos
  implementarlo con este grado de generalidad.
\end{itemize}

\hypertarget{resultados-de-este-algoritmo}{%
\subsubsection{Resultados de Este
Algoritmo}\label{resultados-de-este-algoritmo}}

\texttt{\{r\}\ kable(head(tabla\_rolg))}

\hypertarget{analisis-por-discriminantes}{%
\subsection{Análisis por
Discriminantes}\label{analisis-por-discriminantes}}

La implementación de estas funciones es bastante más sencilla,

\hypertarget{discrimnantes-lineales}{%
\subsubsection{Discrimnantes Lineales}\label{discrimnantes-lineales}}

\hypertarget{discriminantes-cuadraticos}{%
\subsubsection{Discriminantes
Cuadráticos}\label{discriminantes-cuadraticos}}

\hypertarget{analisis-por-los-k-vecinos-mas-cercanos}{%
\subsection{Análisis por los K Vecinos Más
Cercanos}\label{analisis-por-los-k-vecinos-mas-cercanos}}

\hypertarget{evaluacion-realizada}{%
\section{Evaluación Realizada}\label{evaluacion-realizada}}

\hypertarget{conclusiones}{%
\section{Conclusiones}\label{conclusiones}}


\end{document}
